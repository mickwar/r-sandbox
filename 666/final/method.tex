\section{Methods}

Random forests are constructed using an ensemble of trees from bootstrap samples. In this section I describe how to grow random forests and the rationale behind them. The details come from \cite{segal:2011} and \cite{james:2013}.

\subsection{Growing a single tree}

Tree-based methods involve dividing the predictor variables $X_1,\ldots,X_p$ into $J$ distinct and non-overlapping regions $R_1,\ldots,R_J$. These regions may be any shape but are most often hyperrectangles formed by recursive binary splitting. The resulting regions are called terminal nodes, or leaves, and contain a subset of observations not found in any other terminal node. 

Let $t$ be the set of observations within the terminal node being considered for splitting and $s=(v, c)$ denote the variable $v$ and cutoff $c$ as the corresponding split in the predictor space. That is, at split $s$ we separate the observations $i\in t$ into ``left'' and ``right'' groups, $t_L=\{i:x_{vi}\leq c, i\in t\}$ and $t_R=\{i:x_{vi} > c, i\in t\}$. Then grow a tree according to the following algorithm:
\begin{enumerate}
\item Have every observation begin in the same node.
\item For every terminal node $t$ and every split $s$ compute $\phi(s, t)$.
\item Create two new terminal nodes $t_L$ and $t_R$ at split $s$ when $\phi(s, t)$ is optimized. The split node $t$ is now termed a parent node.
\item Repeat steps 2 and 3 until some stopping rule is met.%; for example, until no node contains more than five observations.
\end{enumerate}
This algorithm requires that the predictor variables either be continuous, binary , or ordered categorical measurements. For unordered categories (discrete variables where one value is no ``greater'' than another), splits are considered at every possible separation of the factor levels into two distinct groups.

The function $\phi(s, t)$ is some optimization function that results in the most homogenous children nodes. Homogeneity is assessed via some node impurity measure. In the regression case we may let
\begin{eqnarray}
\phi(s, t) = SS(t) - SS(t_L) - SS(t_R)
\label{reg_phi}
\end{eqnarray}
and would thus want to maximize $\phi(s, t)$ for choosing the next split. Pure nodes, those containing closely related observations, will have minimum sums of squares $SS(t_L) + SS(t_R)$. The multivariate analog of sums of squares is given by
\begin{eqnarray}
SS(t) = \sum_{i\in t}(y_i-\mu(t))^\top V^{-1} (y_i-\mu(t))
\label{reg_ss}
\end{eqnarray}
where the response $y_i$ is a vector of length $q$, $\mu(t)$ is the mean response of every observation in node $t$, and the $q\times q$ matrix $V$ is a weighting ``covariance'' matrix. When $V=I_q$, the impurity of a node is the sum of squared Euclidean distances. When $q=1$ and $V=1$, (\ref{reg_ss}) reduces to $SS(t) = \sum_{i\in t}(y_i-\mu(t))^2$, the univariate sum of the squares.

For $K$ categorical responses, a common impurity measure is the Gini index defined by
\begin{eqnarray}
G(t) = \sum_{k=1}^K\hat{p}_{tk}(1-\hat{p}_{tk})
\label{cla_gini}
\end{eqnarray}
where $\hat{p}_{tk}$ is the proportion of observations in class $k$ found in node $t$. Smaller $G(t)$ indicate purer nodes, and $G(t)$ is close to zero when all of the $\hat{p}_{tk}$'s are close to zero or one. Growing trees using the Gini index would maximize $\phi(s, t)=G(t) - G(t_L) - G(t_R)$. 

Alternatives to the Gini index include the classification error rate, using the most common class in the region as the prediction, and cross-entropy defined as
\[D(t) = -\sum_{k=1}^K\hat{p}_{tk}\mathrm{log}(\hat{p}_{tk}).\]
The Gini index and cross-entropy are very similar numerically and both emphasize node purity, whereas classification error rate emphasizes prediction accuracy.

A common stopping rule (step 4 in the algorithm) is to stop splitting nodes when each node has less observations than some pre-determined quantity. Another rule might be to place a lower limit on the node size, placing a constraint on the tree size.

\subsection{Random forests}

A random forest is constructed by growing $T$ trees on a boostrapped sample (discussed in section 2.3) while only using $m\ll p$ variables to perform a node split. The predictions across all $T$ trees are averaged. For each tree, step 2 of the algorithm in section 2.1 is modified so a random sample of $m$ predictors are considered for splitting. This serves to decorrelated the trees.

For a sequence $X_1,\ldots,X_n$ of random variables, the following is true:
\begin{eqnarray}
\mathrm{Var}\left(\frac{1}{n}\sum_{i=1}^nX_i\right) = \frac{1}{n^2}\sum_{i=1}^n\mathrm{Var}(X_i) + \frac{2}{n^2}\sum_{i=1}^{n-1}\sum_{j=i+1}^n\mathrm{Cov}(X_i,X_j).
\label{var}
\end{eqnarray}
When $X_1,\ldots,X_n$ are distributed such that $\mathrm{Var}(X_i)=\sigma^2$ and $\mathrm{Cor}(X_i, X_j)=\rho$, then (\ref{var}) reduces to
\begin{eqnarray}
\mathrm{Var}\left(\frac{1}{n}\sum_{i=1}^nX_i\right) = \frac{\sigma^2}{n} + \frac{n-1}{n}\rho\sigma^2
\end{eqnarray}
where $\rho$ is the average correlation of distinct variables. Hence, for correlated random variables, taking an average will reduce variance, but the decrease is hampered by the positive correlation. When trees are grown on bootstrap samples, there will typically be some dominating variables that show up most often and possibly other variables never being used for a split. This causes the trees to be positively correlated.

By forcing a random sample of $m\ll p$ predictors to be used in a split will allow those less important variables to show up more often and the more important ones to show up less than they normally would. Thus the trees will be less correlated and the predictive accuracy of the random forest will increase. Typical choices for $m$ include $m=p/3$, $m=\sqrt{p}$, and $m=\log_2(p + 1)$.

\subsection{Bootstrapping}

To obtain a \emph{bootstrap} sample, randomly draw $n$ observations (\emph{with} replacement) from a data set. On average, $n/3$ of the observations will not be part of the bootstrap sample. These are termed \emph{out-of-bag} (OOB) and are used as a test set. Predictions are made on each OOB observation for each of the $T$ trees. We will thus have about $T/3$ predictions for observation $i$, and an overall prediction for $i$ can be made using the average of the predicted responses.

\subsection{Variable importance}

Since we no longer have a single tree, interpretation is much more difficult. However, it is still possible to measure the ``importance'' of a variable: how influential a variable is on dividing the predictor space to improve predictability.

We compute the importance for a variable by averaging the decrease in (\ref{reg_ss}) or (\ref{cla_gini}) caused by splitting the variable. Influential variables will have the greatest decreases in these impurity measures while less importance ones will have minimally decreased the impurity.
