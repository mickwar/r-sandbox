\section{Example: movie preferences}

The methods in section 2 are applied to a movie preference data set. A willing subject was asked two questions per movie: (1) Did you like the movie? and (2) Did you think it was well-executed? The list of movies came from the subject's personal history of movies watched. A ``yes'' response to question one indicates the participant generally liked the movie. A ``yes'' to the second question means the participant felt that the movie was well-made (regardless of whether the movie was liked) in terms of acting, cinematography, soundtrack, and other general qualities of a movie. There are thus four possible categories:
\begin{enumerate}
\item Liked, and well-executed
\item Liked, but not well-executed
\item Disliked, but well-executed
\item Disliked, and not well-executed
\end{enumerate}
As is expected, there are few movies (only two in this dataset) that the subject liked but did not think were well-made. We remove those movies from the data set and consider only the movies in classes 1, 3, and 4.

Movie attributes are Rotten Tomato scores (\texttt{rot\_crit}), Kids in Mind ratings (violence \texttt{kids\_V}, sexual \texttt{kids\_S}, and profanity \texttt{kids\_P}), year of release (\texttt{year}), MPAA rating (\texttt{G}, \texttt{PG}, and \texttt{PG13}), and an indicator for whether the movie was animated or not (\texttt{live\_action}). Some movies do not have a Rotten Tomatoes score and others do not have Kids in Mind ratings; these movies were also removed from the data set. We are left with $n=71$ observations.

We randomly set aside 40 observations for a training set and the remaining 31 as a test set. A large number of trees are grown on bootstrap samples of the training set where at each node a random sample of $m=\sqrt{9}=3$ predictors are chosen to split the node.

\begin{figure}
\begin{center}
\includegraphics[scale=0.4]{figs/forest_error.pdf}
\caption{Classification error rates for each group.}
\label{error}
\end{center}
\end{figure}

\begin{figure}
\begin{center}
\includegraphics[scale=0.4]{figs/importance.pdf}
\caption{Variables ordered by importance (from top to bottom) as measured by the mean decrease in the Gini index.}
\label{import}
\end{center}
\end{figure}

\begin{table}
\begin{center}
\begin{tabular}{lr|rrr|r}
     & \multicolumn{1}{l}{}   & \multicolumn{3}{l}{Predicted} \\
     &    &  1 & 3 & 4 & Error \\ \cline{2-6}
     & 1  & 24 & 1 & 2 & 0.11 \\
True & 3  &  3 & 0 & 1 & 1.00 \\
     & 4  &  7 & 0 & 2 & 0.77 \\
\end{tabular}
\caption{Confusion matrix}
\end{center}
\end{table}
