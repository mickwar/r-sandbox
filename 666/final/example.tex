\section{Example: movie preferences}

The methods in section 2 are applied to a movie preference data set. A willing subject was asked two questions per movie: (1) Did you like the movie? and (2) Did you think it was well-executed? The list of movies came from the subject's personal history of movies watched. A ``yes'' response to question one indicates the participant generally liked the movie. A ``yes'' to the second question means the participant felt that the movie was well-made (regardless of whether the movie was liked) in terms of acting, cinematography, soundtrack, and other general qualities of a movie. There are thus four possible categories:
\begin{enumerate}
\item Liked, and well-executed
\item Liked, but not well-executed
\item Disliked, but well-executed
\item Disliked, and not well-executed
\end{enumerate}
As is expected, there are few movies (only two in this dataset) that were liked but the subject did not think were well-made. We remove those movies from the data set and consider only the movies in classes 1, 3, and 4.
