\section{Tests on Historical Means}

For the olives in regions two and four we want to test whether the means of the acid measurements differ significantly from their historical averages. We test 
\begin{eqnarray*}
H_{02}:\m{\mu}_{02}=(1300,120,265,7310,820,45,65,28)^\top \\
H_{04}:\m{\mu}_{04}=(1230,105,275,7360,830,41,75,38)^\top
\end{eqnarray*}
where $H_{02}$ and $H_{04}$ are the null hypotheses for tests and regions two and four, respectively.

We will use Hotelling's $T^2$ test statistic (a multivariate analogue of the univariate $t$ test statistic) defined as
\[ T^2 = n(\overline{\m{x}}-\m{\mu}_0)^\top\m{S}^{-1}(\overline{\m{x}}-\m{\mu}_0) \]
to test for significant departures from the hypothesized mean $\m{\mu}_0$. Large differences in the mean of any variable from the historical mean will result in a large $T^2$ statistic. We can convert $T^2$ into an $F$ distribution via
\[ \frac{\nu -p+1}{\nu p}T^2 = F \sim F_{p,\nu-p+1}. \]
where $\nu=n-1$ degrees of freedom and $p=8$ acids.

Table \ref{history} shows the test statistics and $p$-values for each region's null hypothesis.
\begin{table}[H]
\begin{center}
\begin{tabular}{l | rrr}
Region & \multicolumn{1}{l}{$T^2$}  & \multicolumn{1}{l}{$F$}    & \multicolumn{1}{l}{$p$-value} \\ \hline \hline
     2 & 10.119 & 1.103  & 0.377     \\
     4 & 20.183 & 2.018  & 0.081     \\
\end{tabular}
\caption{Historical means comparison test results.}
\label{history}
\end{center}
\end{table}
