% Write in Present Tense
% Change first two lines: we have evidence against multivariate normality, but
% not FOR.

\section{Assessing Multivariate Normality}

In this section we use both graphical and numerical tests to diagnose multivariate normality in the data. We consider the univariate and bivariate distribution of each variable, quantile-quantile (Q-Q) plots, and tests on skewness and kurtosis. Violations in any of these areas is evidence against the assumption of multivariate normality.




% When individual columns of data are not univariate or bivariate normal, we can
% conclude that the data are not multivariate normal. The converse, however, is
% not true. The data do not appear normal from the univariate plots of their
% histograms. But, the transformed data appear to be approximately univariate
% and bivariate normal, with the exception of some outliers. Therefore, to
% conclude whether the data are multivariate normal, we must do some other
% formal test. Here, we choose to calculate skewness and kurtosis to help us
% determine whether the data are multivariate normal.
% % What are the outliers? Can I put a plot here / refer to a plot here?

