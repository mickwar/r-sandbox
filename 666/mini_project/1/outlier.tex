\section{Unusual Observations}
%Outliers have different effects on the data, such as changing the mean, variance, or correlation. Analytical tests have been created for outliers that can slightly, positively shift the mean,  shift the mean in either direction, or positively shift the variance. This shifting is called $slippage$. Graphical methods are used as well to detect outliers. We discuss both here applied to this data.

% If you want to include something abotu nonnormmality using visualization.
% Do we suppose that there is mean slippage?
%For the analytical methods, a test on kurtosis is helpful to identify outliers. As mentioned above, the kurtosis statistics were highly significant. The locally best test for mean slippage is based on the sample kurtosis previously defined, meaning that the among all tests invariant to transformations in the data, $b_{2,p}$ is most powerful for detecting shifts in the mean vector.

%This test doubles as a check for multivariate normality and for the presence of outliers. Since we have rejected the null hypothesis that the data are multivariately normally distributed, we assume that they were caused by the more extreme $D_i^2$ values. This leads us to believe that one or more of the following numbers may have been mis-keyed and should be investigated: (1) observation 176, stearic acid; (2) observation 3, palmitoleic acid; (3) observation 40; palmitic acid.These three observations were chosen because of their post-transformed $D_i^2$ values of 26.9, 27.1 and 88.4 respectively. Justification for why we chose these specific acids is provided later.
Particularly high standardized distance $D_i^2$ values lead us to believe that one or more of the following observations may have been mis-keyed and should be investigated: (1) observation 40; palmitic acid; (2) observation 3, palmitoleic acid; and (3) observation 176, stearic acid. These three observations were chosen because of their post-transformed $D_i^2$ values of 88.4, 27.1, and 26.9, respectively. Justification for why we chose these specific acids is provided later.

To identify these observations visually, the data were plotted using both a grand tour in a statistical program for visualization, Ggobi, as well as the bivariate plots of Figure \ref{pairs_both}. A grand tour is a continuous series of two-dimensional projections on the $p$-variate data. Nonnormality and outliers can be detected if their two-dimensional projection shows ``any deviation from elliptical clouds of points'' (p. 107, R\&C). We saw repeated evidence that the data deviated from the elliptical cloud. These three observations (40, 3, and 176) appeared frequently in the grand tour as disrupting the elliptical nature of the data for the specified acid.

Also, when looking at a bivariate plot between palmitic and oleic, it becomes clearer that observation 40 was a strong outlier caused by either oleic or palmitic. By its position, it is weakening the correlation of the data. Furthermore, the Q-Q plot shows these three observations as being highly deviant from normality in Figure \ref{qq_trans}, which means that they could be outliers.

%In trying to identify which acids were influential in the observation's distance, we further used the grand tour visualization to identify the key components which were influential in that observation being an outlier. Again, looking at 2-D projections of the data, these specific acids repeatedly arose as having strong components when the observations appeared to have outlier-type behavior.
