\documentclass[12pt]{article}

\usepackage{amsmath}
\usepackage{bm}

\newcommand{\m}[1]{\mathbf{\bm{#1}}}
\newcommand{\R}{I\hspace{-4.4pt}R}

\newcommand{\E}{\mathrm{E}}
\newcommand{\var}{\mathrm{var}}

% Brief summary of the paper (highlight important parts as you
% would in a literature review, where this paper is a main source
% for your work -- not everything, just important details).

% Comment on how the paper applies to the class (ideas for each
% paper are listed with each citation, although you are not limited
% to these suggestions).

% Extend the results of the paper to yoru own analysis.


\begin{document}

\noindent Mickey Warner
\bigskip

\noindent Stat 637 -- Mini Project \# 1

\section*{Introduction}

\noindent In this report, we review Zeger \emph{et al.}'s 1988 paper titled \emph{Models for Longitudinal Data: A Generalized Estimating Equation Approach}. The authors consider two classes of models: subject-specific (SS) and population average (PA). The focus of a particular problem will determine which of the two models to fit.
\bigskip

\noindent Subject-specific models are used when the response for an individual is the focus. These models taken on the form of a mixed generalized linear model. For each subject $i$, the response $y_{it}$ at time $t$ is assumed to satisfy
\begin{eqnarray}
h(\E[y_{it}|\m{b}_i]) = h(u_{it}) &=& \m{x}_{it}^\top\m{\beta} + \m{z}_{it}^\top\m{b}_i \\
\var(y_{it}|\m{b}_i) &=& g(u_{it})\cdot \phi
\end{eqnarray}
where $\m{b}_i$ is an independent observation from some distribution, $F$, $\m{x}_{it}$ is a $p\times 1$ vector of fixed covariates at time $t$, $\m{z}_{it}$ is $q\times 1$, and $t=1,\ldots,n_i$ and $i=1,\ldots,K$. The functions $h$ and $g$ are referred to as the ``link'' and ``variance'' functions, respectively.


\end{document}
