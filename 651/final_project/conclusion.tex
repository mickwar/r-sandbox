\section{Conclusion}

With only a few field trials, we have used Gaussian processes to calibrate a computer simulator of mass spectrometry experiments. As part of the model, we estimate sensitivity measures to determine which inputs have a significant contribution to the output of the simulator. We have shown that the model fits the data.

Combining emulator predictions with the discrepancy between simulator and reality has improved the predictive power of the simulator. Predictions of the chemical process may be made using the computer simulator adjusted for the discrepancy bias.

Most importantly, we have been able to calculator posterior distributions for the chemical reaction rates. The chemical process occurring in the mass spectrometry yields highly complex chemical reactions, making the reaction rates difficult or impossible to solve for. The use of a computer simulator has provided additional understanding of the underlying chemical process.
