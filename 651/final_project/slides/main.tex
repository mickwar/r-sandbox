\documentclass[mathserif, 11pt, t]{beamer}

\let\Tiny=\tiny
\usepackage{xcolor}
\usepackage{pdfpages}
\usepackage{graphicx}
\usepackage{float}
\usepackage{enumitem}
\usepackage{natbib}
\usepackage{amsmath}
\usepackage{bm}
\geometry{vmargin=0.5in}

%colors
\definecolor{blue}{rgb}{0.05, 0.05, 0.90}
\definecolor{bluegreen}{rgb}{0.05, 0.2, 0.6}

%commands
\newcommand{\lra}{\longrightarrow}
\newcommand{\ra}{\rightarrow}
\newcommand{\m}[1]{\mathbf{\bm{#1}}}
\renewcommand{\frametitle}[1]{\bigskip\textcolor{blue}{%
    \LARGE{\textbf{#1}}}\vspace{0.15cm}\newline}
\renewcommand{\subtitle}[1]{\vspace{0.45cm}\textcolor{bluegreen}{
    {\textbf{#1}}}\vspace{0.15cm}\newline}
\newcommand{\tlb}[1]{\large{\textbf{#1}}}

%slide colors
\pagecolor{blue!40}

\begin{document}

%%% begin title frame
\begin{center}
\ \\ [-0.5in]
\vfill
\bigskip
\bigskip
\bigskip
\bigskip
\bigskip

\begin{LARGE}
\begin{center}
Gaussian Process Modeling of Modern Mass Spectrometry Computer Experimental Data
\end{center}
\end{LARGE}
\vfill

\begin{center}
Mickey Warner
\end{center}
\vfill
21 March 2015
\bigskip
\bigskip
\bigskip
\vfill
\ \\ [-0.5in]
\end{center}
%%% end title frame

\begin{frame}
\subtitle{VENDAMS}
\begin{itemize}[label={$\cdot$}]
\item Variable electron and neutral density attachment mass spectrometry
\smallskip
\item Estimate rates for complex chemical reactions \[A+B\lra C+D\]
\item Measures relative abundances of anions: \[0\leq y_i \leq 1,~~~\sum_{i=1}^{n_y} y_i=1\]
\item Computer simulations used for supplemental data
\end{itemize}
\end{frame}

\begin{frame}
\subtitle{Field trials and computer simulations}
\begin{center}
\includegraphics[scale=0.30]{figs/dat_combine.pdf}
\end{center}
\end{frame}

\begin{frame}
\subtitle{Statistical model}
At initial electron concentration $\m{x}_i$, relative abundances $\m{y}_i$ are modeled according to
\[\mathrm{logit}(\m{y}_i) = \eta(\m{x}_i, \m{\theta}) + \delta(\m{x}_i) + \epsilon_i\]

Mean zero Gaussian processes are put on $\eta(\cdot, \cdot)$ and $\delta(\cdot)$, with product covariance functions having exponential form
\[ \mathrm{C}((\m{x},\m{t}),(\m{x}',\m{t}'); \m{\rho}_\eta) = \frac{1}{\lambda_\eta}\prod_{k=1}^{p_x}\rho_{\eta,k}^{4(x_k-x_k')^2}\times\prod_{k=1}^{p_t}\rho_{\eta,p_x+k}^{4(t_k-t_k')^2} \]
Gamma priors on $\m{\lambda}$, beta priors on $\m{\rho}$, and uniform on $\m{\theta}$. More details found in Higdon, et. al. (2008).
\end{frame}

\begin{frame}
\subtitle{Trace plots}
\vspace{-0.5cm}
\begin{center}
\includegraphics[scale=0.25]{figs/br_trace.pdf}
\end{center}
\end{frame}

\begin{frame}
\subtitle{Kolmogorov-Smirnov}
\begin{center}
\includegraphics[scale=0.25]{../figs/ks.pdf}

$p$-value $=0.37$
\end{center}
\end{frame}

\begin{frame}
\subtitle{Posterior distributions}
\begin{center}
\includegraphics[scale=0.25]{figs/br_boxplot.pdf}~
\includegraphics[scale=0.22]{figs/br_theta.pdf}
\end{center}
\end{frame}

\begin{frame}
\subtitle{Posterior predictions}
\vspace{-1.0cm}
\begin{center}
\includegraphics[scale=0.25]{figs/br_pred_all.pdf}
\end{center}
Left: statistical emulator for the simulator

Center: discrepancy between reality and simulator

Right: discrepancy-adjusted (calibrated) predictions

\end{frame}

\begin{frame}
\subtitle{Conclusion}
\vspace{-0.5cm}
\begin{itemize}[label={$\cdot$}]
\item Simulations, adjusted for field trials, predict the process well
\item The model fits
\item Reaction rates are estimated
\end{itemize}

\subtitle{Future work}
\vspace{-0.5cm}
\begin{itemize}[label={$\cdot$}]
\item Other ways to deal with contraints in the response
\end{itemize}
\end{frame}


\end{document}
