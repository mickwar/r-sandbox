\documentclass[12pt]{article}

\usepackage{graphicx}
\usepackage{float}
\usepackage{amsmath}
\usepackage{bm}
\usepackage[font=footnotesize]{caption}
\usepackage{subcaption}
\usepackage[margin=1.25in]{geometry}

\newcommand{\m}[1]{\mathbf{\bm{#1}}}
\newcommand{\R}{I\hspace{-4.4pt}R}

\begin{document}

\noindent Mickey Warner

\noindent Stat 651 -- Mini Project 3

\section{Bootstrap importance sampling}

\subsection{Likelihood and priors}

\noindent We model the teacher ratings using a truncated normal, which has density

\begin{eqnarray*}
f_*(y|\mu,\sigma^2,a,b) = \frac{\frac{1}{\sigma}\phi(\frac{y-\mu}{\sigma})}{\Phi(\frac{b-\mu}{\sigma})-\Phi(\frac{a-\mu}{\sigma})} & -\infty < \mu < +\infty; & \sigma > 0; \\
& -\infty \leq a < b \leq +\infty; & a < y < b,
\end{eqnarray*}

\noindent where $\phi(\cdot)$ is the standard normal density and $\Phi(\cdot)$ is the standard normal c.d.f. We fix $a=1$ and $b=7$. We assign the parameters $\mu$ and $\sigma^2$ to be independent $\mathcal{N}(5, 10^2)$ and $\mathcal{IG}(2.5, 1.5)$, respectively.

\subsection{Importance function}

\noindent We will use the multivariate $t$ distribution, denoted $t_\nu(\m{\mu}, \m{\Sigma})$, as our importance function, which has density

\begin{eqnarray*}
I(\m{x}) = \frac{\Gamma[(\nu+p)/2]\nu^{-p/2}\pi^{-p/2}|\m{\Sigma}|^{-1/2}}{\Gamma(\nu/2)[1+\frac{1}{\nu}(\m{x}-\m{\mu})^\top\m{\Sigma}^{-1}(\m{x}-\m{\mu})]^{(\nu+p)/2}}; && \m{x}, \m{\mu} \in \R^p; \nu>0; \\
&& \m{\Sigma}~\mathrm{pos.~def.~in~} \R^{p\times p}
\end{eqnarray*}

\noindent where, after trial and error, we fix $\nu = 4$, $\m{\Sigma} = \left(\begin{array}{ll} 0.0522 & 0.0370 \\ 0.0370 & 0.0501 \\ \end{array}\right)$, and $\m{\mu}=(5.78, 0.31)^\top$ which is about the mode of the posterior. There are $p=2$ parameters. This function has a similar shape as the joint posterior of the parameters. Having low degrees of freedom ($\nu=4$) means we have heavy tails and so we should have no problems in low probability regions.

\subsection{Expectations}

\begin{eqnarray*}
\mathrm{E}(\mu,\sigma^2|\m{y}) &=& (5.808,~0.393) \\
\mathrm{Var}(\mu, \sigma^2|\m{y}) &=& \left(\begin{array}{ll} 0.0257 & 0.0112 \\ 0.0112 & 0.0275 \\ \end{array}\right) \\
\sqrt{\mathrm{Var}(\mu, \sigma^2|\m{y})} &=& (0.160,~0.165)
\end{eqnarray*}

\subsection{Joint posterior}

\begin{figure}[H]
    \centering
    \includegraphics[scale=0.49]{figs/boot_joint.pdf}
    \caption*{}
\end{figure}


\subsection{Posterior predictive}

\begin{figure}[H]
    \centering
    \includegraphics[scale=0.49]{figs/boot_pred.pdf}
    \caption*{}
\end{figure}

\noindent The probability of scoring a 5 or better is $0.897$.

\section{Metropolis-Hastings algorithm}

\begin{figure}[H]
    \centering
    \includegraphics[scale=0.50]{figs/mh_post.pdf}
    \caption*{}
\end{figure}

\noindent The figure on the left is of the unnormalized target density (dotted blue) with the density estimation of $10,000$ draws from the Metropolis algorithm. The figure on the right has the target function scaled to some guesstimated value so the draws match up.
\bigskip

\begin{center}
\begin{tabular}{ll}
\multicolumn{2}{c}{Metropolis Settings} \\ \hline \smallskip
nburn            & 5000 \\ \smallskip
nmcmc            & 10000 \\ \smallskip
$p(x, y)$ & $y\sim\mathcal{N}(x, \sigma_\mathrm{cand}^2)$ \\ \smallskip
$\sigma_{\mathrm{cand}} $ & 5.77 \\ \smallskip
acceptance rate  & 0.267 \\
\end{tabular}
\end{center}

\newpage

\section{Gibbs and M-H}

\subsection{Likelihood and priors}

\noindent We use the three parameter Weibull to model a single observation

\begin{eqnarray*}
f_*(y_i|\alpha, \beta, \mu) = \frac{\beta}{\alpha}(y_i-\mu)^{\beta-1}\exp\left(-\frac{(y_i-\mu)^\beta}{\alpha}\right) & -\infty < \mu < +\infty; & y_i \geq \mu; \\
& \alpha, \beta > 0.
\end{eqnarray*}

\noindent Since the experts believe that no ball bearing should fail under $10$ revolutions (in $10^6$ units), we fix $\mu=10$. % For notational convenience, we omit the $\mu$ parameter hereafter, but the code and anaylsis incorporates this information.
\bigskip

\noindent The likelihood is given by

\begin{eqnarray*}
L(\m{y}|\alpha, \beta) &=& \prod_{i=1}^n\frac{\beta}{\alpha}(y_i-\mu)^{\beta-1}\exp\left(-\frac{(y_i-\mu)^\beta}{\alpha}\right)  \\
 &=& \beta^n\alpha^{-n}\left(\prod (y_i-\mu)\right)^{\beta-1}\exp\left(-\frac{\sum (y_i-\mu)^\beta}{\alpha}\right)
\end{eqnarray*}

\noindent When holding $\beta$ constant, we recongize the likelihood to have the form of an inverse gamma (under the rate parametrization) in terms of $\alpha$. We assume $\alpha$ follows an $\mathcal{IG}(a, b)$. It can be shown that the distribution of $\alpha|\beta,\m{y}$ follows an $\mathcal{IG}(n+a,b+\sum(y_i-\mu)^\beta)$. The inverse gamma is decently flexible, so we feel that this is a reasonable assumption to make on $\alpha$ since we ought to be able to capture its behavior. Also, the conjugate prior gives us some computational convenience.
\bigskip

\noindent There is no conjugate prior for $\beta$. We assume $\beta$ is distributed $\Gamma(c, d)$, indepedent of $\alpha$. This (as does $\alpha$) has the right support.
\bigskip

\noindent We are told the ball bearings should average about $50-70$ revolutions. After playing around with some parameters in the Weibull, we find that a $\mathrm{Weibull}(800, 1.5)$ comes close to this prior knowledge. We choose $a=4$ and $b=2500$ which leads to a prior expectation of around $800$ and prior standard deviation of about $500$, which should capture that $800$ in the Weibull. We choose $c=3$ and $d=2$ which has prior mean and standard deviation of $1.5$ and $0.75$, respectively.

\subsection{Posterior distributions}

\begin{eqnarray*}
\mathrm{E}(\alpha, \beta|\m{y}) &=& (893.9,~1.578) \\
\mathrm{Var}(\alpha, \beta|\m{y}) &=& \left(\begin{array}{rr} 282800.0000 & 51.6600 \\ 51.6600 & 0.0134 \\ \end{array}\right) \\
\end{eqnarray*}

\begin{figure}[H]
    \centering
    \includegraphics[scale=0.50]{figs/gibb_joint.pdf}
\end{figure}

\begin{figure}[H]
    \centering
    \includegraphics[scale=0.50]{figs/gibb_marginal.pdf}
\end{figure}

\begin{center}
\begin{tabular}{l|rr}
\multicolumn{3}{c}{97\% HPD Intervals} \\
Param & \multicolumn{1}{l}{Lower} & \multicolumn{1}{l}{Upper} \\ \hline
$\alpha$  & 227.00 & 2152.00 \\
$\beta$   & 1.34 & 1.83 \\
\end{tabular}
\end{center}

\subsection{Posterior predictive distribution}

\begin{figure}[H]
    \centering
    \includegraphics[scale=0.50]{figs/gibb_pred.pdf}
\end{figure}

\noindent The probability of a ball being superior is $0.127$. That is, $Pr(Y_{n+1} > 120|\m{y}) \approx 0.127$.

\subsection{MCMC settings}

\begin{center}
\begin{tabular}{ll}
\multicolumn{2}{c}{Metropolis Settings} \\ \hline \smallskip
nburn            & 20000 \\ \smallskip
nmcmc            & 100000 \\ \smallskip
$p(x, y)$ & $y\sim\mathcal{N}(x, \sigma_\mathrm{cand}^2)$ \\ \smallskip
$\sigma_{\mathrm{cand}} $ & 0.218 \\ \smallskip
acceptance rate  & 0.253 \\
\end{tabular}
\end{center}

\end{document}
