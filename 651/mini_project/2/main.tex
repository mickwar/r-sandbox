\documentclass[12pt]{article}

\usepackage{graphicx}
\usepackage{float}
\usepackage{amsmath}
\usepackage{bm}
\usepackage[font=footnotesize]{caption}
\newcommand{\m}[1]{\mathbf{\bm{#1}}}

\begin{document}

\noindent Mickey Warner

\noindent Stat 651 -- Mini Project 2

\section{Rejection Sampling}

\subsection{Likelihood}

\noindent The measurements we are analyzing are teacher ratings (averaged across students) for a sample of faculty members. It seems reasonable to assume that the university hires its faculty according to similar standards and expectations. We thus assume each observation comes from the same distribution.
\bigskip

\noindent Further, since we are working with means we appeal to the Central Limit Theorem and propose the truncated normal (the data is bounded from 1 to 7) as the distribution of a single observation. This has density

\begin{eqnarray*}
f_*(y|\mu,\sigma^2,a,b) = \frac{\frac{1}{\sigma}\phi(\frac{y-\mu}{\sigma})}{\Phi(\frac{b-\mu}{\sigma})-\Phi(\frac{a-\mu}{\sigma})} & -\infty < \mu < +\infty; & \sigma > 0; \\
& -\infty \leq a < b \leq +\infty; & a < y < b, \\
\end{eqnarray*}

\noindent where $\phi(\cdot)$ is the density of the standard normal distribution,$\Phi(\cdot)$ is the c.d.f. of the standard normal, and $a$ and $b$ are the lower and upper bounds. We fix $a=1$ and $b=7$ because of our knowledge of the data. The likelihood function is

\[L(\m{y}|\m{\theta}) = \prod_{i=1}^n f_*(y_i|\m{\theta}) \]

\noindent for $\m{y}=(y_1,\ldots,y_n)^\top$ and $\m{\theta}=(\mu, \sigma^2)$ with $n=23$.

\subsection{Prior on $(\mu, \sigma^2)$}

\noindent We assume $\mu$ and $\sigma^2$ are indepedent so we may write

\[ \pi(\mu, \sigma^2) = \pi(\mu)\pi(\sigma^2). \]

\noindent This will simplify computations and there isn't any evidence to suggest nonindependence nor to necessitate the use of a hierarchical specification for $\mu$ and $\sigma^2$.
\bigskip

\noindent We let $\mu\sim\mathcal{N}(5, 10^2)$ and $\sigma^2\sim\mathcal{IG}(2.5, 1.5)$ using the inverse gamma with the following parametrization

\[ \pi(\sigma^2) = \frac{b^a}{\Gamma(a)}(\sigma^2)^{-a-1}\exp(-b/\sigma^2);~~~ a,b,\sigma^2>0. \]

\noindent The hyperpriors on $\mu$ result in a distribution centered around 5, which is to say we think on average the faculty would have a score of 5. The prior on $\sigma^2$ has a mean of 1, variance of 1.5, and the central 95\% density interval is $(0.234, 3.612)$. These both represent fairly noninformative priors.

\subsection{The Envelope Distribution}

\begin{figure}[H]
\begin{center}
\includegraphics[scale=0.5]{figs/contour.pdf}
\end{center}
\caption{okay}
\end{figure}


\section{Importance Sampling}

\section{Approach Comparison}

\end{document}
